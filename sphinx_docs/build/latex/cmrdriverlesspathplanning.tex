%% Generated by Sphinx.
\def\sphinxdocclass{report}
\documentclass[letterpaper,10pt,english]{sphinxmanual}
\ifdefined\pdfpxdimen
   \let\sphinxpxdimen\pdfpxdimen\else\newdimen\sphinxpxdimen
\fi \sphinxpxdimen=.75bp\relax
\ifdefined\pdfimageresolution
    \pdfimageresolution= \numexpr \dimexpr1in\relax/\sphinxpxdimen\relax
\fi
%% let collapsible pdf bookmarks panel have high depth per default
\PassOptionsToPackage{bookmarksdepth=5}{hyperref}

\PassOptionsToPackage{booktabs}{sphinx}
\PassOptionsToPackage{colorrows}{sphinx}

\PassOptionsToPackage{warn}{textcomp}
\usepackage[utf8]{inputenc}
\ifdefined\DeclareUnicodeCharacter
% support both utf8 and utf8x syntaxes
  \ifdefined\DeclareUnicodeCharacterAsOptional
    \def\sphinxDUC#1{\DeclareUnicodeCharacter{"#1}}
  \else
    \let\sphinxDUC\DeclareUnicodeCharacter
  \fi
  \sphinxDUC{00A0}{\nobreakspace}
  \sphinxDUC{2500}{\sphinxunichar{2500}}
  \sphinxDUC{2502}{\sphinxunichar{2502}}
  \sphinxDUC{2514}{\sphinxunichar{2514}}
  \sphinxDUC{251C}{\sphinxunichar{251C}}
  \sphinxDUC{2572}{\textbackslash}
\fi
\usepackage{cmap}
\usepackage[T1]{fontenc}
\usepackage{amsmath,amssymb,amstext}
\usepackage{babel}



\usepackage{tgtermes}
\usepackage{tgheros}
\renewcommand{\ttdefault}{txtt}



\usepackage[Bjarne]{fncychap}
\usepackage{sphinx}

\fvset{fontsize=auto}
\usepackage{geometry}


% Include hyperref last.
\usepackage{hyperref}
% Fix anchor placement for figures with captions.
\usepackage{hypcap}% it must be loaded after hyperref.
% Set up styles of URL: it should be placed after hyperref.
\urlstyle{same}

\addto\captionsenglish{\renewcommand{\contentsname}{Contents:}}

\usepackage{sphinxmessages}
\setcounter{tocdepth}{1}



\title{CMR Driverless Path Planning}
\date{Jul 18, 2025}
\release{1.0}
\author{Carnegie Mellon Racing}
\newcommand{\sphinxlogo}{\vbox{}}
\renewcommand{\releasename}{Release}
\makeindex
\begin{document}

\ifdefined\shorthandoff
  \ifnum\catcode`\=\string=\active\shorthandoff{=}\fi
  \ifnum\catcode`\"=\active\shorthandoff{"}\fi
\fi

\pagestyle{empty}
\sphinxmaketitle
\pagestyle{plain}
\sphinxtableofcontents
\pagestyle{normal}
\phantomsection\label{\detokenize{index::doc}}



\chapter{What is Path Planning in charge of?}
\label{\detokenize{index:what-is-path-planning-in-charge-of}}
\sphinxAtStartPar
Path planning is responsible for determining the route the car should follow to navigate the track. It has two main responsibilities:
\begin{enumerate}
\sphinxsetlistlabels{\arabic}{enumi}{enumii}{}{.}%
\item {} 
\sphinxAtStartPar
\sphinxstylestrong{Midline Calculation}

\item {} 
\sphinxAtStartPar
{\hyperref[\detokenize{glossary:term-SLAM}]{\sphinxtermref{\DUrole{xref}{\DUrole{std}{\DUrole{std-term}{SLAM}}}}}} \sphinxstylestrong{(Simultaneous Localization and Mapping)}

\end{enumerate}


\chapter{What data does it receive?}
\label{\detokenize{index:what-data-does-it-receive}}
\sphinxAtStartPar
Path planning receives cone detections (positions of blue and yellow cones) from the perceptions pipeline. These cone positions are observed using LiDAR and processed in real\sphinxhyphen{}time. It also receives velocity, orientation, and position in the form of {\hyperref[\detokenize{glossary:term-Twist}]{\sphinxtermref{\DUrole{xref}{\DUrole{std}{\DUrole{std-term}{twist}}}}}}, {\hyperref[\detokenize{glossary:term-Quaternion}]{\sphinxtermref{\DUrole{xref}{\DUrole{std}{\DUrole{std-term}{quaternion}}}}}}, and {\hyperref[\detokenize{glossary:term-Pose}]{\sphinxtermref{\DUrole{xref}{\DUrole{std}{\DUrole{std-term}{pose}}}}}}. These readings are filtered and come from the {\hyperref[\detokenize{glossary:term-IMU}]{\sphinxtermref{\DUrole{xref}{\DUrole{std}{\DUrole{std-term}{IMU}}}}}} and {\hyperref[\detokenize{glossary:term-GPS}]{\sphinxtermref{\DUrole{xref}{\DUrole{std}{\DUrole{std-term}{GPS}}}}}}.

\begin{sphinxadmonition}{note}{Note:}\begin{itemize}
\item {} 
\sphinxAtStartPar
{\hyperref[\detokenize{glossary:term-Twist}]{\sphinxtermref{\DUrole{xref}{\DUrole{std}{\DUrole{std-term}{twist}}}}}} gives us linear and angular velocity

\item {} 
\sphinxAtStartPar
{\hyperref[\detokenize{glossary:term-Quaternion}]{\sphinxtermref{\DUrole{xref}{\DUrole{std}{\DUrole{std-term}{quaternion}}}}}} tells us yaw, pitch, and roll

\item {} 
\sphinxAtStartPar
{\hyperref[\detokenize{glossary:term-Pose}]{\sphinxtermref{\DUrole{xref}{\DUrole{std}{\DUrole{std-term}{pose}}}}}} gives us position and orientation

\end{itemize}

\sphinxAtStartPar
These components combined allow us to reason about the motion of the car as well as the relative position of cones.
\end{sphinxadmonition}


\chapter{What does it output, and to what system?}
\label{\detokenize{index:what-does-it-output-and-to-what-system}}
\sphinxAtStartPar
The output is a set of path waypoints (from midline calculation) and a {\hyperref[\detokenize{glossary:term-SLAM}]{\sphinxtermref{\DUrole{xref}{\DUrole{std}{\DUrole{std-term}{SLAM}}}}}}\sphinxhyphen{}generated map. The path is sent to the controls pipeline, which uses it to generate control actions for the car. The map is used in later laps to improve navigation.

\sphinxstepscope


\section{Implementation}
\label{\detokenize{implementation:implementation}}\label{\detokenize{implementation::doc}}

\subsection{SLAM Nodes}
\label{\detokenize{implementation:slam-nodes}}
\sphinxAtStartPar
There are three primary nodes for {\hyperref[\detokenize{glossary:term-SLAM}]{\sphinxtermref{\DUrole{xref}{\DUrole{std}{\DUrole{std-term}{SLAM}}}}}}.
\begin{enumerate}
\sphinxsetlistlabels{\arabic}{enumi}{enumii}{}{.}%
\item {} 
\sphinxAtStartPar
\sphinxstylestrong{real\_data\_slam\_node\_gps}

\item {} 
\sphinxAtStartPar
\sphinxstylestrong{real\_data\_slam\_node\_no\_gps}

\item {} 
\sphinxAtStartPar
\sphinxstylestrong{controls\_sim\_slam\_node}

\end{enumerate}

\sphinxAtStartPar
The first two nodes are used to run on real data depending on whether we have {\hyperref[\detokenize{glossary:term-GPS}]{\sphinxtermref{\DUrole{xref}{\DUrole{std}{\DUrole{std-term}{GPS}}}}}} or not. The last node is used to test our {\hyperref[\detokenize{glossary:term-SLAM}]{\sphinxtermref{\DUrole{xref}{\DUrole{std}{\DUrole{std-term}{SLAM}}}}}} implementation in\sphinxhyphen{}house with a {\hyperref[\detokenize{glossary:term-Rosbag}]{\sphinxtermref{\DUrole{xref}{\DUrole{std}{\DUrole{std-term}{rosbag}}}}}} recorded on our in\sphinxhyphen{}house controls sim.


\subsection{Execution and Data flow}
\label{\detokenize{implementation:execution-and-data-flow}}
\sphinxAtStartPar
\sphinxstylestrong{Where does} {\hyperref[\detokenize{glossary:term-iSAM2}]{\sphinxtermref{\DUrole{xref}{\DUrole{std}{\DUrole{std-term}{iSAM2}}}}}} \sphinxstylestrong{run (CPU/GPU)?}

\sphinxAtStartPar
The {\hyperref[\detokenize{glossary:term-iSAM2}]{\sphinxtermref{\DUrole{xref}{\DUrole{std}{\DUrole{std-term}{iSAM2}}}}}} {\hyperref[\detokenize{glossary:term-SLAM}]{\sphinxtermref{\DUrole{xref}{\DUrole{std}{\DUrole{std-term}{SLAM}}}}}} implementation runs entirely on the CPU. It is written in C++ and uses {\hyperref[\detokenize{glossary:term-GTSAM}]{\sphinxtermref{\DUrole{xref}{\DUrole{std}{\DUrole{std-term}{GTSAM}}}}}}, a CPU\sphinxhyphen{}optimized factor graph library. No GPU acceleration is used, as {\hyperref[\detokenize{glossary:term-iSAM2}]{\sphinxtermref{\DUrole{xref}{\DUrole{std}{\DUrole{std-term}{iSAM2}}}}}} is designed for incremental updates that are efficient enough for real time execution on modern multi\sphinxhyphen{}core CPUs. Our {\hyperref[\detokenize{glossary:term-SLAM}]{\sphinxtermref{\DUrole{xref}{\DUrole{std}{\DUrole{std-term}{SLAM}}}}}} nodes run within a {\hyperref[\detokenize{glossary:term-ROS-2-Node}]{\sphinxtermref{\DUrole{xref}{\DUrole{std}{\DUrole{std-term}{ROS 2 Node}}}}}} written in C++ and leverage threading when available (through {\hyperref[\detokenize{glossary:term-TBB}]{\sphinxtermref{\DUrole{xref}{\DUrole{std}{\DUrole{std-term}{TBB}}}}}}), although much of the computation remains serial due to the incremental nature of the updates.

\sphinxAtStartPar
\sphinxstylestrong{What is the data flow?}
\begin{enumerate}
\sphinxsetlistlabels{\arabic}{enumi}{enumii}{}{.}%
\item {} 
\sphinxAtStartPar
Input:

\end{enumerate}
\begin{itemize}
\item {} 
\sphinxAtStartPar
Cone observations from the perceptions pipeline (2D positions of blue/yellow cones)

\item {} 
\sphinxAtStartPar
{\hyperref[\detokenize{glossary:term-IMU}]{\sphinxtermref{\DUrole{xref}{\DUrole{std}{\DUrole{std-term}{IMU}}}}}} and velocity estimates

\item {} 
\sphinxAtStartPar
{\hyperref[\detokenize{glossary:term-Pose}]{\sphinxtermref{\DUrole{xref}{\DUrole{std}{\DUrole{std-term}{Pose}}}}}} estimates from {\hyperref[\detokenize{glossary:term-GPS}]{\sphinxtermref{\DUrole{xref}{\DUrole{std}{\DUrole{std-term}{GPS}}}}}} (optional)

\end{itemize}
\begin{enumerate}
\sphinxsetlistlabels{\arabic}{enumi}{enumii}{}{.}%
\setcounter{enumi}{1}
\item {} 
\sphinxAtStartPar
Processing:

\end{enumerate}
\begin{itemize}
\item {} 
\sphinxAtStartPar
Each cone observation is added as a {\hyperref[\detokenize{glossary:term-Bearing-range-factor}]{\sphinxtermref{\DUrole{xref}{\DUrole{std}{\DUrole{std-term}{bearing\sphinxhyphen{}range factor}}}}}} connecting the vehicle {\hyperref[\detokenize{glossary:term-Pose}]{\sphinxtermref{\DUrole{xref}{\DUrole{std}{\DUrole{std-term}{pose}}}}}} to cone {\hyperref[\detokenize{glossary:term-Landmark}]{\sphinxtermref{\DUrole{xref}{\DUrole{std}{\DUrole{std-term}{landmark}}}}}}s

\item {} 
\sphinxAtStartPar
The current vehicle pose (if available) is added as a new variable in the {\hyperref[\detokenize{glossary:term-Factor-Graph}]{\sphinxtermref{\DUrole{xref}{\DUrole{std}{\DUrole{std-term}{Factor Graph}}}}}}

\item {} 
\sphinxAtStartPar
Motion priors are added between sequential poses (using {\hyperref[\detokenize{glossary:term-IMU}]{\sphinxtermref{\DUrole{xref}{\DUrole{std}{\DUrole{std-term}{IMU}}}}}}/velocity)

\item {} 
\sphinxAtStartPar
This information is fed into {\hyperref[\detokenize{glossary:term-iSAM2}]{\sphinxtermref{\DUrole{xref}{\DUrole{std}{\DUrole{std-term}{iSAM2}}}}}} which incrementally updates the estimates of all previous poses and {\hyperref[\detokenize{glossary:term-Landmark}]{\sphinxtermref{\DUrole{xref}{\DUrole{std}{\DUrole{std-term}{landmark}}}}}} (cone) positions

\end{itemize}
\begin{enumerate}
\sphinxsetlistlabels{\arabic}{enumi}{enumii}{}{.}%
\setcounter{enumi}{2}
\item {} 
\sphinxAtStartPar
Output:

\end{enumerate}
\begin{itemize}
\item {} 
\sphinxAtStartPar
Estimated vehicle {\hyperref[\detokenize{glossary:term-Pose}]{\sphinxtermref{\DUrole{xref}{\DUrole{std}{\DUrole{std-term}{pose}}}}}} (2D position and orientation) at each time step

\item {} 
\sphinxAtStartPar
Updated cone map as global positions of all {\hyperref[\detokenize{glossary:term-Landmark}]{\sphinxtermref{\DUrole{xref}{\DUrole{std}{\DUrole{std-term}{landmark}}}}}}s (cones)

\end{itemize}

\sphinxstepscope


\section{Explainers}
\label{\detokenize{explainers:explainers}}\label{\detokenize{explainers::doc}}

\subsection{How is Midline Generated?}
\label{\detokenize{explainers:how-is-midline-generated}}
\sphinxAtStartPar
Our midline calculation relies on Support Vector Machines ({\hyperref[\detokenize{glossary:term-SVM}]{\sphinxtermref{\DUrole{xref}{\DUrole{std}{\DUrole{std-term}{SVM}}}}}}s). Using the cones received from our lidar, the {\hyperref[\detokenize{glossary:term-SVM}]{\sphinxtermref{\DUrole{xref}{\DUrole{std}{\DUrole{std-term}{SVM}}}}}} generates a decision boundary using a cubic polynomial kernel. This decision boundary is converted into a series of points that represents the path the car should take.


\subsection{Why iSAM2?}
\label{\detokenize{explainers:why-isam2}}
\sphinxAtStartPar
We chose this algorithm after considering 1. {\hyperref[\detokenize{glossary:term-iSAM2}]{\sphinxtermref{\DUrole{xref}{\DUrole{std}{\DUrole{std-term}{iSAM2}}}}}} is capable of incrementally optimizing its estimates for previous cone position estimates and {\hyperref[\detokenize{glossary:term-Pose}]{\sphinxtermref{\DUrole{xref}{\DUrole{std}{\DUrole{std-term}{pose}}}}}} estimates, as opposed to optimizing only at loop closure, 2. {\hyperref[\detokenize{glossary:term-iSAM2}]{\sphinxtermref{\DUrole{xref}{\DUrole{std}{\DUrole{std-term}{iSAM2}}}}}}’s performance made it a clear choice for {\hyperref[\detokenize{glossary:term-SLAM}]{\sphinxtermref{\DUrole{xref}{\DUrole{std}{\DUrole{std-term}{SLAM}}}}}}, and 3. we are able to work closely with the author of {\hyperref[\detokenize{glossary:term-iSAM2}]{\sphinxtermref{\DUrole{xref}{\DUrole{std}{\DUrole{std-term}{iSAM2}}}}}}, Professor Michael Kaess. Thus, we would like to take this opportunity to thank Professor Michael Kaess for dedicating his time and efforts to assisting our implementation of {\hyperref[\detokenize{glossary:term-SLAM}]{\sphinxtermref{\DUrole{xref}{\DUrole{std}{\DUrole{std-term}{SLAM}}}}}}.


\subsection{What is iSAM2 and Factor Graph SLAM?}
\label{\detokenize{explainers:what-is-isam2-and-factor-graph-slam}}
\sphinxAtStartPar
{\hyperref[\detokenize{glossary:term-iSAM2}]{\sphinxtermref{\DUrole{xref}{\DUrole{std}{\DUrole{std-term}{iSAM2}}}}}} (Incremental Smoothing and Mapping) is a {\hyperref[\detokenize{glossary:term-SLAM}]{\sphinxtermref{\DUrole{xref}{\DUrole{std}{\DUrole{std-term}{SLAM}}}}}} (Simultaneous Localization and Mapping) algorithm used to construct a map of the track from the car’s position and the cones observed around the track. {\hyperref[\detokenize{glossary:term-iSAM2}]{\sphinxtermref{\DUrole{xref}{\DUrole{std}{\DUrole{std-term}{iSAM2}}}}}} does this by constructing and optimizing a {\hyperref[\detokenize{glossary:term-Factor-Graph}]{\sphinxtermref{\DUrole{xref}{\DUrole{std}{\DUrole{std-term}{Factor Graph}}}}}} containing {\hyperref[\detokenize{glossary:term-Variable-Node}]{\sphinxtermref{\DUrole{xref}{\DUrole{std}{\DUrole{std-term}{Variable Node}}}}}} s (which represent either {\hyperref[\detokenize{glossary:term-Landmark}]{\sphinxtermref{\DUrole{xref}{\DUrole{std}{\DUrole{std-term}{landmark}}}}}} poses or car poses) and {\hyperref[\detokenize{glossary:term-Factor-Node}]{\sphinxtermref{\DUrole{xref}{\DUrole{std}{\DUrole{std-term}{Factor Node}}}}}} s.


\subsection{The Factor Graph}
\label{\detokenize{explainers:the-factor-graph}}
\sphinxAtStartPar
As stated previously, {\hyperref[\detokenize{glossary:term-iSAM2}]{\sphinxtermref{\DUrole{xref}{\DUrole{std}{\DUrole{std-term}{iSAM2}}}}}} relies on a {\hyperref[\detokenize{glossary:term-Factor-Graph}]{\sphinxtermref{\DUrole{xref}{\DUrole{std}{\DUrole{std-term}{Factor Graph}}}}}} containing {\hyperref[\detokenize{glossary:term-Variable-Node}]{\sphinxtermref{\DUrole{xref}{\DUrole{std}{\DUrole{std-term}{Variable Node}}}}}}s and {\hyperref[\detokenize{glossary:term-Factor-Node}]{\sphinxtermref{\DUrole{xref}{\DUrole{std}{\DUrole{std-term}{Factor Node}}}}}}s. {\hyperref[\detokenize{glossary:term-Factor-Node}]{\sphinxtermref{\DUrole{xref}{\DUrole{std}{\DUrole{std-term}{Factor Node}}}}}}s, akin to labeled edges between the {\hyperref[\detokenize{glossary:term-Variable-Node}]{\sphinxtermref{\DUrole{xref}{\DUrole{std}{\DUrole{std-term}{Variable Node}}}}}}s, represent a {\hyperref[\detokenize{glossary:term-Joint-probability-distribution}]{\sphinxtermref{\DUrole{xref}{\DUrole{std}{\DUrole{std-term}{joint probability distribution}}}}}} on the {\hyperref[\detokenize{glossary:term-Variable-Node}]{\sphinxtermref{\DUrole{xref}{\DUrole{std}{\DUrole{std-term}{Variable Node}}}}}}s connected to it. This {\hyperref[\detokenize{glossary:term-Joint-probability-distribution}]{\sphinxtermref{\DUrole{xref}{\DUrole{std}{\DUrole{std-term}{joint probability distribution}}}}}} represents how certain {\hyperref[\detokenize{glossary:term-iSAM2}]{\sphinxtermref{\DUrole{xref}{\DUrole{std}{\DUrole{std-term}{iSAM2}}}}}} is of the corresponding variables’ positions.

\noindent{\hspace*{\fill}\sphinxincludegraphics{{factor_graph}.png}\hspace*{\fill}}

\begin{sphinxadmonition}{note}{Note:}
\sphinxAtStartPar
A {\hyperref[\detokenize{glossary:term-Variable-Node}]{\sphinxtermref{\DUrole{xref}{\DUrole{std}{\DUrole{std-term}{Variable Node}}}}}} cannot be adjacent to another {\hyperref[\detokenize{glossary:term-Variable-Node}]{\sphinxtermref{\DUrole{xref}{\DUrole{std}{\DUrole{std-term}{Variable Node}}}}}} and a {\hyperref[\detokenize{glossary:term-Factor-Node}]{\sphinxtermref{\DUrole{xref}{\DUrole{std}{\DUrole{std-term}{Factor Node}}}}}} cannot be adjacent to another {\hyperref[\detokenize{glossary:term-Factor-Node}]{\sphinxtermref{\DUrole{xref}{\DUrole{std}{\DUrole{std-term}{Factor Node}}}}}}. Blue nodes represent {\hyperref[\detokenize{glossary:term-Variable-Node}]{\sphinxtermref{\DUrole{xref}{\DUrole{std}{\DUrole{std-term}{Variable Node}}}}}}s (X variable nodes represent car poses, L variable nodes represent {\hyperref[\detokenize{glossary:term-Landmark}]{\sphinxtermref{\DUrole{xref}{\DUrole{std}{\DUrole{std-term}{landmark}}}}}} positions). Gray nodes represent {\hyperref[\detokenize{glossary:term-Factor-Node}]{\sphinxtermref{\DUrole{xref}{\DUrole{std}{\DUrole{std-term}{Factor Node}}}}}}s. {\hyperref[\detokenize{glossary:term-Factor-Node}]{\sphinxtermref{\DUrole{xref}{\DUrole{std}{\DUrole{std-term}{Factor Node}}}}}} \(f_{0}\) is called a {\hyperref[\detokenize{glossary:term-Prior-Factor}]{\sphinxtermref{\DUrole{xref}{\DUrole{std}{\DUrole{std-term}{Prior Factor}}}}}} node; {\hyperref[\detokenize{glossary:term-Prior-Factor}]{\sphinxtermref{\DUrole{xref}{\DUrole{std}{\DUrole{std-term}{prior factor}}}}}} nodes are added to the first {\hyperref[\detokenize{glossary:term-Pose}]{\sphinxtermref{\DUrole{xref}{\DUrole{std}{\DUrole{std-term}{pose}}}}}} and sometimes the first {\hyperref[\detokenize{glossary:term-Landmark}]{\sphinxtermref{\DUrole{xref}{\DUrole{std}{\DUrole{std-term}{landmark}}}}}} for {\hyperref[\detokenize{glossary:term-iSAM2}]{\sphinxtermref{\DUrole{xref}{\DUrole{std}{\DUrole{std-term}{iSAM2}}}}}} to use as reference when localizing and mapping future {\hyperref[\detokenize{glossary:term-Pose}]{\sphinxtermref{\DUrole{xref}{\DUrole{std}{\DUrole{std-term}{pose}}}}}}s and {\hyperref[\detokenize{glossary:term-Landmark}]{\sphinxtermref{\DUrole{xref}{\DUrole{std}{\DUrole{std-term}{landmark}}}}}}s.
\end{sphinxadmonition}

\sphinxAtStartPar
For example, observe how in Figure 1, {\hyperref[\detokenize{glossary:term-Factor-Node}]{\sphinxtermref{\DUrole{xref}{\DUrole{std}{\DUrole{std-term}{Factor Node}}}}}} \(f_{1}\) is connected to {\hyperref[\detokenize{glossary:term-Variable-Node}]{\sphinxtermref{\DUrole{xref}{\DUrole{std}{\DUrole{std-term}{Variable Node}}}}}} \(x_{0}\), representing the first car {\hyperref[\detokenize{glossary:term-Pose}]{\sphinxtermref{\DUrole{xref}{\DUrole{std}{\DUrole{std-term}{pose}}}}}}, and \(l_{0}\), representing the first {\hyperref[\detokenize{glossary:term-Landmark}]{\sphinxtermref{\DUrole{xref}{\DUrole{std}{\DUrole{std-term}{landmark}}}}}}. {\hyperref[\detokenize{glossary:term-Factor-Node}]{\sphinxtermref{\DUrole{xref}{\DUrole{std}{\DUrole{std-term}{Factor Node}}}}}} \(f_{1}\) represents a {\hyperref[\detokenize{glossary:term-Joint-probabilistic-distribution}]{\sphinxtermref{\DUrole{xref}{\DUrole{std}{\DUrole{std-term}{joint probabilistic distribution}}}}}} function over \(x_{0}\) and \(l_{0}\), which indicates how certain {\hyperref[\detokenize{glossary:term-iSAM2}]{\sphinxtermref{\DUrole{xref}{\DUrole{std}{\DUrole{std-term}{iSAM2}}}}}} is of the positions for \(x_{0}\) and \(l_{0}\). Altogether, the entire {\hyperref[\detokenize{glossary:term-Factor-Graph}]{\sphinxtermref{\DUrole{xref}{\DUrole{std}{\DUrole{std-term}{Factor Graph}}}}}} represents a {\hyperref[\detokenize{glossary:term-Joint-probabilistic-distribution}]{\sphinxtermref{\DUrole{xref}{\DUrole{std}{\DUrole{std-term}{joint probabilistic distribution}}}}}} function F on all {\hyperref[\detokenize{glossary:term-Landmark}]{\sphinxtermref{\DUrole{xref}{\DUrole{std}{\DUrole{std-term}{landmark}}}}}} positions and car {\hyperref[\detokenize{glossary:term-Pose}]{\sphinxtermref{\DUrole{xref}{\DUrole{std}{\DUrole{std-term}{pose}}}}}}s. This function F is equal to the product of all factors \(f_{n}\), the {\hyperref[\detokenize{glossary:term-Joint-probabilistic-distribution}]{\sphinxtermref{\DUrole{xref}{\DUrole{std}{\DUrole{std-term}{joint probabilistic distribution}}}}}} function represented by each {\hyperref[\detokenize{glossary:term-Factor-Node}]{\sphinxtermref{\DUrole{xref}{\DUrole{std}{\DUrole{std-term}{Factor Node}}}}}} in the graph.

\sphinxAtStartPar
\sphinxstylestrong{Goal with respect to the Factor Graph}
The goal of {\hyperref[\detokenize{glossary:term-iSAM2}]{\sphinxtermref{\DUrole{xref}{\DUrole{std}{\DUrole{std-term}{iSAM2}}}}}} is to maximize the {\hyperref[\detokenize{glossary:term-Joint-probabilistic-distribution}]{\sphinxtermref{\DUrole{xref}{\DUrole{std}{\DUrole{std-term}{joint probabilistic distribution}}}}}} function F by maximizing its factors. Intuitively, {\hyperref[\detokenize{glossary:term-iSAM2}]{\sphinxtermref{\DUrole{xref}{\DUrole{std}{\DUrole{std-term}{iSAM2}}}}}} is seeking to maximize its certainty of {\hyperref[\detokenize{glossary:term-Landmark}]{\sphinxtermref{\DUrole{xref}{\DUrole{std}{\DUrole{std-term}{landmark}}}}}} positions and car {\hyperref[\detokenize{glossary:term-Pose}]{\sphinxtermref{\DUrole{xref}{\DUrole{std}{\DUrole{std-term}{pose}}}}}}s by updating its estimates for car {\hyperref[\detokenize{glossary:term-Pose}]{\sphinxtermref{\DUrole{xref}{\DUrole{std}{\DUrole{std-term}{pose}}}}}}s and {\hyperref[\detokenize{glossary:term-Landmark}]{\sphinxtermref{\DUrole{xref}{\DUrole{std}{\DUrole{std-term}{landmark}}}}}} positions over time (with the help of incoming observations). Considering the previous example, {\hyperref[\detokenize{glossary:term-iSAM2}]{\sphinxtermref{\DUrole{xref}{\DUrole{std}{\DUrole{std-term}{iSAM2}}}}}} can maximize this function F by maximizing the {\hyperref[\detokenize{glossary:term-Joint-probability-distribution}]{\sphinxtermref{\DUrole{xref}{\DUrole{std}{\DUrole{std-term}{joint probability distribution}}}}}} function represented by {\hyperref[\detokenize{glossary:term-Factor-Node}]{\sphinxtermref{\DUrole{xref}{\DUrole{std}{\DUrole{std-term}{Factor Node}}}}}} \(f_{1}\).

\sphinxAtStartPar
\sphinxstylestrong{Implementation}

\noindent{\hspace*{\fill}\sphinxincludegraphics{{observation_step}.png}\hspace*{\fill}}

\sphinxAtStartPar
The {\hyperref[\detokenize{glossary:term-iSAM2}]{\sphinxtermref{\DUrole{xref}{\DUrole{std}{\DUrole{std-term}{iSAM2}}}}}} node first parses the cones received by perceptions into separate vectors by color. This vector of observed cones and other {\hyperref[\detokenize{glossary:term-Odometry}]{\sphinxtermref{\DUrole{xref}{\DUrole{std}{\DUrole{std-term}{Odometry}}}}}} information is used to update the {\hyperref[\detokenize{glossary:term-iSAM2}]{\sphinxtermref{\DUrole{xref}{\DUrole{std}{\DUrole{std-term}{iSAM2}}}}}} model. Using the {\hyperref[\detokenize{glossary:term-Odometry}]{\sphinxtermref{\DUrole{xref}{\DUrole{std}{\DUrole{std-term}{Odometry}}}}}} information, the {\hyperref[\detokenize{glossary:term-iSAM2}]{\sphinxtermref{\DUrole{xref}{\DUrole{std}{\DUrole{std-term}{iSAM2}}}}}} node predicts the car’s current {\hyperref[\detokenize{glossary:term-Pose}]{\sphinxtermref{\DUrole{xref}{\DUrole{std}{\DUrole{std-term}{pose}}}}}} using the received {\hyperref[\detokenize{glossary:term-Odometry}]{\sphinxtermref{\DUrole{xref}{\DUrole{std}{\DUrole{std-term}{Odometry}}}}}} information. {\hyperref[\detokenize{glossary:term-Variable-Node}]{\sphinxtermref{\DUrole{xref}{\DUrole{std}{\DUrole{std-term}{Variable Node}}}}}} \(x_{n}\), representing the car {\hyperref[\detokenize{glossary:term-Pose}]{\sphinxtermref{\DUrole{xref}{\DUrole{std}{\DUrole{std-term}{pose}}}}}} at the current time stamp, is added alongside a {\hyperref[\detokenize{glossary:term-Factor-Node}]{\sphinxtermref{\DUrole{xref}{\DUrole{std}{\DUrole{std-term}{Factor Node}}}}}} connecting \(x_{n}\) to \(x_{n-1}\), the {\hyperref[\detokenize{glossary:term-Variable-Node}]{\sphinxtermref{\DUrole{xref}{\DUrole{std}{\DUrole{std-term}{Variable Node}}}}}} representing the previous car {\hyperref[\detokenize{glossary:term-Pose}]{\sphinxtermref{\DUrole{xref}{\DUrole{std}{\DUrole{std-term}{pose}}}}}}.

\noindent{\hspace*{\fill}\sphinxincludegraphics{{data_association}.png}\hspace*{\fill}}

\sphinxAtStartPar
After determining the car {\hyperref[\detokenize{glossary:term-Pose}]{\sphinxtermref{\DUrole{xref}{\DUrole{std}{\DUrole{std-term}{pose}}}}}}, {\hyperref[\detokenize{glossary:term-Data-Association}]{\sphinxtermref{\DUrole{xref}{\DUrole{std}{\DUrole{std-term}{Data Association}}}}}} is performed on the cones observed at the current timestamp to determine which of the observed cones are new. To perform this {\hyperref[\detokenize{glossary:term-Data-Association}]{\sphinxtermref{\DUrole{xref}{\DUrole{std}{\DUrole{std-term}{Data Association}}}}}}, the {\hyperref[\detokenize{glossary:term-Mahalanobis-Distance}]{\sphinxtermref{\DUrole{xref}{\DUrole{std}{\DUrole{std-term}{Mahalanobis Distance}}}}}} is calculated between one observed cone, and all {\hyperref[\detokenize{glossary:term-iSAM2}]{\sphinxtermref{\DUrole{xref}{\DUrole{std}{\DUrole{std-term}{iSAM2}}}}}} estimates for the previously seen cones. Intuitively, the {\hyperref[\detokenize{glossary:term-Mahalanobis-Distance}]{\sphinxtermref{\DUrole{xref}{\DUrole{std}{\DUrole{std-term}{Mahalanobis Distance}}}}}} represents how much the observed cone resembles a previously seen cone (the smaller the distance, the more the observed cone resembles the previously seen cone). If the smallest distance is greater than the {\hyperref[\detokenize{glossary:term-Mahalanobis-Distance}]{\sphinxtermref{\DUrole{xref}{\DUrole{std}{\DUrole{std-term}{Mahalanobis Distance}}}}}} Threshold, then the observed cone is a new cone.

\begin{sphinxadmonition}{note}{Note:}
\sphinxAtStartPar
The {\hyperref[\detokenize{glossary:term-Mahalanobis-Distance}]{\sphinxtermref{\DUrole{xref}{\DUrole{std}{\DUrole{std-term}{Mahalanobis Distance}}}}}} threshold is generally found through tuning and trial and error.
\end{sphinxadmonition}

\begin{sphinxadmonition}{note}{Note:}
\sphinxAtStartPar
{\hyperref[\detokenize{glossary:term-Mahalanobis-Distance}]{\sphinxtermref{\DUrole{xref}{\DUrole{std}{\DUrole{std-term}{Mahalanobis Distance}}}}}} is used instead of Euclidean distance because where Euclidean distance can calculate the distance between two points, {\hyperref[\detokenize{glossary:term-Mahalanobis-Distance}]{\sphinxtermref{\DUrole{xref}{\DUrole{std}{\DUrole{std-term}{Mahalanobis Distance}}}}}} can calculate the distance between a point and a distribution. This is important because the cone positions come with uncertainty which is represented by a distribution (See \sphinxhref{https://www.machinelearningplus.com/statistics/mahalanobis-distance/}{more})
\end{sphinxadmonition}

\noindent{\hspace*{\fill}\sphinxincludegraphics{{updated_factor_graph}.png}\hspace*{\fill}}

\sphinxAtStartPar
This process is repeated for all observed cones. Each detected new cone must be added to the {\hyperref[\detokenize{glossary:term-Factor-Graph}]{\sphinxtermref{\DUrole{xref}{\DUrole{std}{\DUrole{std-term}{Factor Graph}}}}}} as a {\hyperref[\detokenize{glossary:term-Variable-Node}]{\sphinxtermref{\DUrole{xref}{\DUrole{std}{\DUrole{std-term}{Variable Node}}}}}} with a {\hyperref[\detokenize{glossary:term-Factor-Node}]{\sphinxtermref{\DUrole{xref}{\DUrole{std}{\DUrole{std-term}{Factor Node}}}}}} connected to \(x_{n}\), the {\hyperref[\detokenize{glossary:term-Variable-Node}]{\sphinxtermref{\DUrole{xref}{\DUrole{std}{\DUrole{std-term}{Variable Node}}}}}} representing the current car {\hyperref[\detokenize{glossary:term-Pose}]{\sphinxtermref{\DUrole{xref}{\DUrole{std}{\DUrole{std-term}{pose}}}}}}.

\sphinxstepscope


\section{Extraneous}
\label{\detokenize{extraneous:extraneous}}\label{\detokenize{extraneous::doc}}

\subsection{Extra Thoughts}
\label{\detokenize{extraneous:extra-thoughts}}
\sphinxAtStartPar
While the current {\hyperref[\detokenize{glossary:term-SLAM}]{\sphinxtermref{\DUrole{xref}{\DUrole{std}{\DUrole{std-term}{SLAM}}}}}} and path planning system is designed to be efficient and robust, several challenges may arise in practice:
\begin{quote}

\sphinxAtStartPar
Sensor Noise and Drift: {\hyperref[\detokenize{glossary:term-GPS}]{\sphinxtermref{\DUrole{xref}{\DUrole{std}{\DUrole{std-term}{GPS}}}}}} and {\hyperref[\detokenize{glossary:term-IMU}]{\sphinxtermref{\DUrole{xref}{\DUrole{std}{\DUrole{std-term}{IMU}}}}}} measurements can be noisy or suffer from drift. This can lead to inaccurate localization or erroneous map updates.

\sphinxAtStartPar
{\hyperref[\detokenize{glossary:term-Data-Association}]{\sphinxtermref{\DUrole{xref}{\DUrole{std}{\DUrole{std-term}{Data Association}}}}}} Ambiguity: Identifying whether a newly observed cone corresponds to a previously seen cone ({\hyperref[\detokenize{glossary:term-Data-Association}]{\sphinxtermref{\DUrole{xref}{\DUrole{std}{\DUrole{std-term}{data association}}}}}}) is challenging in cluttered environments. Misassociations can degrade map quality and cause localization accuracy.

\sphinxAtStartPar
Tuning Sensitivity: {\hyperref[\detokenize{glossary:term-SLAM}]{\sphinxtermref{\DUrole{xref}{\DUrole{std}{\DUrole{std-term}{SLAM}}}}}} performance heavily depends on parameters such as the {\hyperref[\detokenize{glossary:term-Mahalanobis-Distance}]{\sphinxtermref{\DUrole{xref}{\DUrole{std}{\DUrole{std-term}{Mahalanobis distance}}}}}} threshold and noise models. Poorly tuned parameters can result in either missed associations or false positives.

\sphinxAtStartPar
Failure Recovery: If the system becomes mislocalized or the {\hyperref[\detokenize{glossary:term-Factor-Graph}]{\sphinxtermref{\DUrole{xref}{\DUrole{std}{\DUrole{std-term}{factor graph}}}}}} diverges due to accumulated error, recovering gracefully is non\sphinxhyphen{}trivial and may require additional loop closure strategies or reinitialization.

\sphinxAtStartPar
Real World Edge Cases: Lighting conditions, partial occlusions, or uneven terrain can cause discrepancies between perception and ground truth, which affect downstream {\hyperref[\detokenize{glossary:term-SLAM}]{\sphinxtermref{\DUrole{xref}{\DUrole{std}{\DUrole{std-term}{SLAM}}}}}} and planning modules.
\end{quote}

\sphinxstepscope


\section{API Reference}
\label{\detokenize{isam2:api-reference}}\label{\detokenize{isam2::doc}}

\subsection{slamISAM Class}
\label{\detokenize{isam2:slamisam-class}}\index{slam::slamISAM (C++ class)@\spxentry{slam::slamISAM}\spxextra{C++ class}}

\begin{fulllineitems}
\phantomsection\label{\detokenize{isam2:_CPPv4N4slam8slamISAME}}\phantomsection\label{\detokenize{isam2:_CPPv3N4slam8slamISAME}}\phantomsection\label{\detokenize{isam2:_CPPv2N4slam8slamISAME}}\phantomsection\label{\detokenize{isam2:slam::slamISAM}}
\pysigstartsignatures
\pysigstartmultiline
\pysigline
{\phantomsection\label{\detokenize{isam2:classslam_1_1slamISAM}}\DUrole{k}{class}\DUrole{w}{ }\sphinxbfcode{\sphinxupquote{\DUrole{n}{slamISAM}}}}
\pysigstopmultiline
\pysigstopsignatures
\begin{sphinxuseclass}{breathe-sectiondef}\subsubsection*{Public Functions}
\index{slam::slamISAM::slamISAM (C++ function)@\spxentry{slam::slamISAM::slamISAM}\spxextra{C++ function}}

\begin{fulllineitems}
\phantomsection\label{\detokenize{isam2:_CPPv4N4slam8slamISAM8slamISAMENSt8optionalIN6rclcpp6LoggerEEERNSt8optionalIN11yaml_params11NoiseInputsEEE}}\phantomsection\label{\detokenize{isam2:_CPPv3N4slam8slamISAM8slamISAMENSt8optionalIN6rclcpp6LoggerEEERNSt8optionalIN11yaml_params11NoiseInputsEEE}}\phantomsection\label{\detokenize{isam2:_CPPv2N4slam8slamISAM8slamISAMENSt8optionalIN6rclcpp6LoggerEEERNSt8optionalIN11yaml_params11NoiseInputsEEE}}\phantomsection\label{\detokenize{isam2:slam::slamISAM::slamISAM__std::optional:rclcpp::Logger:.std::optional:yaml_params::NoiseInputs:R}}
\pysigstartsignatures
\pysigstartmultiline
\pysiglinewithargsret
{\phantomsection\label{\detokenize{isam2:classslam_1_1slamISAM_1a0f15f59ec41b33eeca84b0399a6429dc}}\sphinxbfcode{\sphinxupquote{\DUrole{n}{slamISAM}}}}
{\DUrole{n}{std}\DUrole{p}{::}\DUrole{n}{optional}\DUrole{p}{\textless{}}\DUrole{n}{rclcpp}\DUrole{p}{::}\DUrole{n}{Logger}\DUrole{p}{\textgreater{}}\DUrole{w}{ }\DUrole{n}{\DUrole{sig-param}{input\_logger}}\sphinxparamcomma \DUrole{n}{std}\DUrole{p}{::}\DUrole{n}{optional}\DUrole{p}{\textless{}}\DUrole{n}{yaml\_params}\DUrole{p}{::}\DUrole{n}{NoiseInputs}\DUrole{p}{\textgreater{}}\DUrole{w}{ }\DUrole{p}{\&}\DUrole{n}{\DUrole{sig-param}{yaml\_noise\_inputs}}}
{}
\pysigstopmultiline
\pysigstopsignatures
\sphinxAtStartPar
Constructor for the {\hyperref[\detokenize{isam2:classslam_1_1slamISAM}]{\sphinxcrossref{\DUrole{std}{\DUrole{std-ref}{slamISAM}}}}} class. 

\sphinxAtStartPar
Initializes noise models, tunable constants, and internal parameters required for iSAM2 SLAM operation. It also sets up the noise models used for different factor types (landmark, odometry, prior, and unary).
\begin{quote}\begin{description}
\sphinxlineitem{Parameters}\begin{itemize}
\item {} 
\sphinxAtStartPar
\sphinxstyleliteralstrong{\sphinxupquote{input\_logger}} \textendash{} Optional ROS 2 logger for debug output 

\item {} 
\sphinxAtStartPar
\sphinxstyleliteralstrong{\sphinxupquote{yaml\_noise\_inputs}} \textendash{} Optional struct containing SLAM tuning parameters 

\end{itemize}

\end{description}\end{quote}

\end{fulllineitems}

\index{slam::slamISAM::slamISAM (C++ function)@\spxentry{slam::slamISAM::slamISAM}\spxextra{C++ function}}

\begin{fulllineitems}
\phantomsection\label{\detokenize{isam2:_CPPv4N4slam8slamISAM8slamISAMEv}}\phantomsection\label{\detokenize{isam2:_CPPv3N4slam8slamISAM8slamISAMEv}}\phantomsection\label{\detokenize{isam2:_CPPv2N4slam8slamISAM8slamISAMEv}}\phantomsection\label{\detokenize{isam2:slam::slamISAM::slamISAM}}
\pysigstartsignatures
\pysigstartmultiline
\pysiglinewithargsret
{\phantomsection\label{\detokenize{isam2:classslam_1_1slamISAM_1a339835603720c6ba0db771b8fd9a8828}}\DUrole{k}{inline}\DUrole{w}{ }\sphinxbfcode{\sphinxupquote{\DUrole{n}{slamISAM}}}}
{}
{}
\pysigstopmultiline
\pysigstopsignatures
\end{fulllineitems}

\index{slam::slamISAM::step (C++ function)@\spxentry{slam::slamISAM::step}\spxextra{C++ function}}

\begin{fulllineitems}
\phantomsection\label{\detokenize{isam2:_CPPv4N4slam8slamISAM4stepENSt8optionalIN5gtsam6Point2EEEdRKNSt6vectorIN5gtsam6Point2EEERKNSt6vectorIN5gtsam6Point2EEERKNSt6vectorIN5gtsam6Point2EEEN5gtsam5Pose2Ed}}\phantomsection\label{\detokenize{isam2:_CPPv3N4slam8slamISAM4stepENSt8optionalIN5gtsam6Point2EEEdRKNSt6vectorIN5gtsam6Point2EEERKNSt6vectorIN5gtsam6Point2EEERKNSt6vectorIN5gtsam6Point2EEEN5gtsam5Pose2Ed}}\phantomsection\label{\detokenize{isam2:_CPPv2N4slam8slamISAM4stepENSt8optionalIN5gtsam6Point2EEEdRKNSt6vectorIN5gtsam6Point2EEERKNSt6vectorIN5gtsam6Point2EEERKNSt6vectorIN5gtsam6Point2EEEN5gtsam5Pose2Ed}}\phantomsection\label{\detokenize{isam2:slam::slamISAM::step__std::optional:gtsam::Point2:.double.std::vector:gtsam::Point2:CR.std::vector:gtsam::Point2:CR.std::vector:gtsam::Point2:CR.gtsam::Pose2.double}}
\pysigstartsignatures
\pysigstartmultiline
\pysiglinewithargsret
{\phantomsection\label{\detokenize{isam2:classslam_1_1slamISAM_1a59469d87c99142bec093ab4a13c9cff5}}\DUrole{n}{slam\_output\_t}\DUrole{w}{ }\sphinxbfcode{\sphinxupquote{\DUrole{n}{step}}}}
{\DUrole{n}{std}\DUrole{p}{::}\DUrole{n}{optional}\DUrole{p}{\textless{}}\DUrole{n}{gtsam}\DUrole{p}{::}\DUrole{n}{Point2}\DUrole{p}{\textgreater{}}\DUrole{w}{ }\DUrole{n}{\DUrole{sig-param}{gps\_opt}}\sphinxparamcomma \DUrole{kt}{double}\DUrole{w}{ }\DUrole{n}{\DUrole{sig-param}{yaw}}\sphinxparamcomma \DUrole{k}{const}\DUrole{w}{ }\DUrole{n}{std}\DUrole{p}{::}\DUrole{n}{vector}\DUrole{p}{\textless{}}\DUrole{n}{gtsam}\DUrole{p}{::}\DUrole{n}{Point2}\DUrole{p}{\textgreater{}}\DUrole{w}{ }\DUrole{p}{\&}\DUrole{n}{\DUrole{sig-param}{cone\_obs\_blue}}\sphinxparamcomma \DUrole{k}{const}\DUrole{w}{ }\DUrole{n}{std}\DUrole{p}{::}\DUrole{n}{vector}\DUrole{p}{\textless{}}\DUrole{n}{gtsam}\DUrole{p}{::}\DUrole{n}{Point2}\DUrole{p}{\textgreater{}}\DUrole{w}{ }\DUrole{p}{\&}\DUrole{n}{\DUrole{sig-param}{cone\_obs\_yellow}}\sphinxparamcomma \DUrole{k}{const}\DUrole{w}{ }\DUrole{n}{std}\DUrole{p}{::}\DUrole{n}{vector}\DUrole{p}{\textless{}}\DUrole{n}{gtsam}\DUrole{p}{::}\DUrole{n}{Point2}\DUrole{p}{\textgreater{}}\DUrole{w}{ }\DUrole{p}{\&}\DUrole{n}{\DUrole{sig-param}{orange\_ref\_cones}}\sphinxparamcomma \DUrole{n}{gtsam}\DUrole{p}{::}\DUrole{n}{Pose2}\DUrole{w}{ }\DUrole{n}{\DUrole{sig-param}{velocity}}\sphinxparamcomma \DUrole{kt}{double}\DUrole{w}{ }\DUrole{n}{\DUrole{sig-param}{dt}}}
{}
\pysigstopmultiline
\pysigstopsignatures
\sphinxAtStartPar
Performs a full SLAM update step using odometry and observed cone landmarks. 

\sphinxAtStartPar
This function updates the SLAM factor graph using new motion data and cone observations. It includes steps such as pose prediction, data association, graph updates with both existing and new landmarks, and optional loop closure detection.
\begin{quote}\begin{description}
\sphinxlineitem{Parameters}\begin{itemize}
\item {} 
\sphinxAtStartPar
\sphinxstyleliteralstrong{\sphinxupquote{gps\_opt}} \textendash{} Optional GPS position of the car 

\item {} 
\sphinxAtStartPar
\sphinxstyleliteralstrong{\sphinxupquote{yaw}} \textendash{} The heading of the car in radians 

\item {} 
\sphinxAtStartPar
\sphinxstyleliteralstrong{\sphinxupquote{cone\_obs\_blue}} \textendash{} The observed blue cones in the local frame 

\item {} 
\sphinxAtStartPar
\sphinxstyleliteralstrong{\sphinxupquote{cone\_obs\_yellow}} \textendash{} The observed yellow cones in the local frame 

\item {} 
\sphinxAtStartPar
\sphinxstyleliteralstrong{\sphinxupquote{orange\_ref\_cones}} \textendash{} The orange reference cones 

\item {} 
\sphinxAtStartPar
\sphinxstyleliteralstrong{\sphinxupquote{velocity}} \textendash{} The car’s motion since the last step as a relative Pose2 

\item {} 
\sphinxAtStartPar
\sphinxstyleliteralstrong{\sphinxupquote{dt}} \textendash{} Time elapsed since the last SLAM step

\end{itemize}

\sphinxlineitem{Returns}
\sphinxAtStartPar
A tuple containing:\begin{itemize}
\item {} 
\sphinxAtStartPar
A vector of the most recent estimated blue cone positions as Points

\item {} 
\sphinxAtStartPar
A vector of the most recent estimated yellow cone positions as Points

\item {} 
\sphinxAtStartPar
A Point representing the current pose of the car 

\end{itemize}


\end{description}\end{quote}

\end{fulllineitems}


\end{sphinxuseclass}
\end{fulllineitems}



\subsection{SLAMEstAndMCov Class}
\label{\detokenize{isam2:slamestandmcov-class}}\index{slam::SLAMEstAndMCov (C++ class)@\spxentry{slam::SLAMEstAndMCov}\spxextra{C++ class}}

\begin{fulllineitems}
\phantomsection\label{\detokenize{isam2:_CPPv4N4slam14SLAMEstAndMCovE}}\phantomsection\label{\detokenize{isam2:_CPPv3N4slam14SLAMEstAndMCovE}}\phantomsection\label{\detokenize{isam2:_CPPv2N4slam14SLAMEstAndMCovE}}\phantomsection\label{\detokenize{isam2:slam::SLAMEstAndMCov}}
\pysigstartsignatures
\pysigstartmultiline
\pysigline
{\phantomsection\label{\detokenize{isam2:classslam_1_1SLAMEstAndMCov}}\DUrole{k}{class}\DUrole{w}{ }\sphinxbfcode{\sphinxupquote{\DUrole{n}{SLAMEstAndMCov}}}}
\pysigstopmultiline
\pysigstopsignatures
\sphinxAtStartPar
Maintains and updates landmark estimates and marginal covariance matrices using iSAM2. 

\sphinxAtStartPar
This class provides a lgihtweight interface around GTSAM’s iSAM2 for managing SLAM estimates and covariance matrices for landmarks (cones), which are continuously updated during the first lap. It supports partial and full recalculation, cone\sphinxhyphen{}based proximity updates, and Mahalanobis distance computation. 

\begin{sphinxuseclass}{breathe-sectiondef}\subsubsection*{Public Functions}
\index{slam::SLAMEstAndMCov::SLAMEstAndMCov (C++ function)@\spxentry{slam::SLAMEstAndMCov::SLAMEstAndMCov}\spxextra{C++ function}}

\begin{fulllineitems}
\phantomsection\label{\detokenize{isam2:_CPPv4N4slam14SLAMEstAndMCov14SLAMEstAndMCovENSt10shared_ptrIN5gtsam5ISAM2EEEPFN5gtsam6SymbolEiENSt6size_tENSt6size_tE}}\phantomsection\label{\detokenize{isam2:_CPPv3N4slam14SLAMEstAndMCov14SLAMEstAndMCovENSt10shared_ptrIN5gtsam5ISAM2EEEPFN5gtsam6SymbolEiENSt6size_tENSt6size_tE}}\phantomsection\label{\detokenize{isam2:_CPPv2N4slam14SLAMEstAndMCov14SLAMEstAndMCovENSt10shared_ptrIN5gtsam5ISAM2EEEPFN5gtsam6SymbolEiENSt6size_tENSt6size_tE}}
\pysigstartsignatures
\pysigstartmultiline
\pysiglinewithargsret
{\phantomsection\label{\detokenize{isam2:classslam_1_1SLAMEstAndMCov_1a84beb5098e3d71eb9fbe802a16cbad23}}\sphinxbfcode{\sphinxupquote{\DUrole{n}{SLAMEstAndMCov}}}}
{\DUrole{n}{std}\DUrole{p}{::}\DUrole{n}{shared\_ptr}\DUrole{p}{\textless{}}\DUrole{n}{gtsam}\DUrole{p}{::}\DUrole{n}{ISAM2}\DUrole{p}{\textgreater{}}\DUrole{w}{ }\DUrole{n}{\DUrole{sig-param}{isam2}}\sphinxparamcomma \DUrole{n}{gtsam}\DUrole{p}{::}\DUrole{n}{Symbol}\DUrole{w}{ }\DUrole{p}{(}\DUrole{p}{*}\DUrole{n}{\DUrole{sig-param}{cone\_key\_fn}}\DUrole{p}{)}\DUrole{p}{(}\DUrole{kt}{int}\DUrole{p}{)}\sphinxparamcomma \DUrole{n}{std}\DUrole{p}{::}\DUrole{n}{size\_t}\DUrole{w}{ }\DUrole{n}{\DUrole{sig-param}{look\_radius}}\sphinxparamcomma \DUrole{n}{std}\DUrole{p}{::}\DUrole{n}{size\_t}\DUrole{w}{ }\DUrole{n}{\DUrole{sig-param}{update\_iterations\_n}}}
{}
\pysigstopmultiline
\pysigstopsignatures
\sphinxAtStartPar
Construct a new {\hyperref[\detokenize{isam2:classslam_1_1SLAMEstAndMCov_1a84beb5098e3d71eb9fbe802a16cbad23}]{\sphinxcrossref{\DUrole{std}{\DUrole{std-ref}{SLAMEstAndMCov::SLAMEstAndMCov}}}}} object. 

\sphinxAtStartPar
Construct a new {\hyperref[\detokenize{isam2:classslam_1_1SLAMEstAndMCov}]{\sphinxcrossref{\DUrole{std}{\DUrole{std-ref}{SLAMEstAndMCov}}}}} object with configuration parameters.
\begin{quote}\begin{description}
\sphinxlineitem{Parameters}\begin{itemize}
\item {} 
\sphinxAtStartPar
\sphinxstyleliteralstrong{\sphinxupquote{isam2}} \textendash{} A shared pointer to the iSAM2 model that will be used for obtaining cone estimates and marginal covariance matrices.

\item {} 
\sphinxAtStartPar
\sphinxstyleliteralstrong{\sphinxupquote{cone\_key\_fn}} \textendash{} A function pointer used for getting the symbol of a cone/landmark of interest. This symbol is used for retrieving cone estimates and the marginal covariance matrix of any cone or landmark given its ID.

\item {} 
\sphinxAtStartPar
\sphinxstyleliteralstrong{\sphinxupquote{look\_radius}} \textendash{} The number of cones to recalculate estimates and marginal covariances for, surrounding some pivot cone ID.

\item {} 
\sphinxAtStartPar
\sphinxstyleliteralstrong{\sphinxupquote{update\_iterations\_n}} \textendash{} The number of iterations to run the iSAM2 update for. 

\end{itemize}

\end{description}\end{quote}

\end{fulllineitems}

\index{slam::SLAMEstAndMCov::SLAMEstAndMCov (C++ function)@\spxentry{slam::SLAMEstAndMCov::SLAMEstAndMCov}\spxextra{C++ function}}

\begin{fulllineitems}
\phantomsection\label{\detokenize{isam2:_CPPv4N4slam14SLAMEstAndMCov14SLAMEstAndMCovEv}}\phantomsection\label{\detokenize{isam2:_CPPv3N4slam14SLAMEstAndMCov14SLAMEstAndMCovEv}}\phantomsection\label{\detokenize{isam2:_CPPv2N4slam14SLAMEstAndMCov14SLAMEstAndMCovEv}}\phantomsection\label{\detokenize{isam2:slam::SLAMEstAndMCov::SLAMEstAndMCov}}
\pysigstartsignatures
\pysigstartmultiline
\pysiglinewithargsret
{\phantomsection\label{\detokenize{isam2:classslam_1_1SLAMEstAndMCov_1a26b44510e1e84abf576a5129eab2cf5b}}\sphinxbfcode{\sphinxupquote{\DUrole{n}{SLAMEstAndMCov}}}}
{}
{}
\pysigstopmultiline
\pysigstopsignatures
\sphinxAtStartPar
Default constructor for {\hyperref[\detokenize{isam2:classslam_1_1SLAMEstAndMCov}]{\sphinxcrossref{\DUrole{std}{\DUrole{std-ref}{SLAMEstAndMCov}}}}}. 

\sphinxAtStartPar
Initializes internal members to deault values 

\end{fulllineitems}

\index{slam::SLAMEstAndMCov::update\_and\_recalculate\_all (C++ function)@\spxentry{slam::SLAMEstAndMCov::update\_and\_recalculate\_all}\spxextra{C++ function}}

\begin{fulllineitems}
\phantomsection\label{\detokenize{isam2:_CPPv4N4slam14SLAMEstAndMCov26update_and_recalculate_allEv}}\phantomsection\label{\detokenize{isam2:_CPPv3N4slam14SLAMEstAndMCov26update_and_recalculate_allEv}}\phantomsection\label{\detokenize{isam2:_CPPv2N4slam14SLAMEstAndMCov26update_and_recalculate_allEv}}\phantomsection\label{\detokenize{isam2:slam::SLAMEstAndMCov::update_and_recalculate_all}}
\pysigstartsignatures
\pysigstartmultiline
\pysiglinewithargsret
{\phantomsection\label{\detokenize{isam2:classslam_1_1SLAMEstAndMCov_1ad23727c5f88a5096c84cd62c2248dd92}}\DUrole{kt}{void}\DUrole{w}{ }\sphinxbfcode{\sphinxupquote{\DUrole{n}{update\_and\_recalculate\_all}}}}
{}
{}
\pysigstopmultiline
\pysigstopsignatures
\sphinxAtStartPar
Recalculates all of the slam estimates and the marginal covariance matrices. 

\end{fulllineitems}

\index{slam::SLAMEstAndMCov::update\_and\_recalculate\_by\_ID (C++ function)@\spxentry{slam::SLAMEstAndMCov::update\_and\_recalculate\_by\_ID}\spxextra{C++ function}}

\begin{fulllineitems}
\phantomsection\label{\detokenize{isam2:_CPPv4N4slam14SLAMEstAndMCov28update_and_recalculate_by_IDERKNSt6vectorINSt6size_tEEE}}\phantomsection\label{\detokenize{isam2:_CPPv3N4slam14SLAMEstAndMCov28update_and_recalculate_by_IDERKNSt6vectorINSt6size_tEEE}}\phantomsection\label{\detokenize{isam2:_CPPv2N4slam14SLAMEstAndMCov28update_and_recalculate_by_IDERKNSt6vectorINSt6size_tEEE}}\phantomsection\label{\detokenize{isam2:slam::SLAMEstAndMCov::update_and_recalculate_by_ID__std::vector:std::s:CR}}
\pysigstartsignatures
\pysigstartmultiline
\pysiglinewithargsret
{\phantomsection\label{\detokenize{isam2:classslam_1_1SLAMEstAndMCov_1a709a83c6627b62538b82211615c8bc92}}\DUrole{kt}{void}\DUrole{w}{ }\sphinxbfcode{\sphinxupquote{\DUrole{n}{update\_and\_recalculate\_by\_ID}}}}
{\DUrole{k}{const}\DUrole{w}{ }\DUrole{n}{std}\DUrole{p}{::}\DUrole{n}{vector}\DUrole{p}{\textless{}}\DUrole{n}{std}\DUrole{p}{::}\DUrole{n}{size\_t}\DUrole{p}{\textgreater{}}\DUrole{w}{ }\DUrole{p}{\&}\DUrole{n}{\DUrole{sig-param}{old\_cone\_ids}}}
{}
\pysigstopmultiline
\pysigstopsignatures
\sphinxAtStartPar
Recalulates the estiamtes and the marginal covariance matrices for the IDs provided in the vector. 
\begin{quote}\begin{description}
\sphinxlineitem{Parameters}
\sphinxAtStartPar
\sphinxstyleliteralstrong{\sphinxupquote{old\_cone\_ids}} \textendash{} The IDs of the cones to recalculate estimates and marginal covariances for 

\end{description}\end{quote}

\end{fulllineitems}

\index{slam::SLAMEstAndMCov::update\_and\_recalculate\_beginning (C++ function)@\spxentry{slam::SLAMEstAndMCov::update\_and\_recalculate\_beginning}\spxextra{C++ function}}

\begin{fulllineitems}
\phantomsection\label{\detokenize{isam2:_CPPv4N4slam14SLAMEstAndMCov32update_and_recalculate_beginningENSt6size_tE}}\phantomsection\label{\detokenize{isam2:_CPPv3N4slam14SLAMEstAndMCov32update_and_recalculate_beginningENSt6size_tE}}\phantomsection\label{\detokenize{isam2:_CPPv2N4slam14SLAMEstAndMCov32update_and_recalculate_beginningENSt6size_tE}}\phantomsection\label{\detokenize{isam2:slam::SLAMEstAndMCov::update_and_recalculate_beginning__std::s}}
\pysigstartsignatures
\pysigstartmultiline
\pysiglinewithargsret
{\phantomsection\label{\detokenize{isam2:classslam_1_1SLAMEstAndMCov_1a9bc194216650da2a29195fa39433e574}}\DUrole{kt}{void}\DUrole{w}{ }\sphinxbfcode{\sphinxupquote{\DUrole{n}{update\_and\_recalculate\_beginning}}}}
{\DUrole{n}{std}\DUrole{p}{::}\DUrole{n}{size\_t}\DUrole{w}{ }\DUrole{n}{\DUrole{sig-param}{num\_start\_cones}}}
{}
\pysigstopmultiline
\pysigstopsignatures
\sphinxAtStartPar
Recalculates the cone estimates and the marginal covariance matrices for the first num\_start\_cones IDs. 
\begin{quote}\begin{description}
\sphinxlineitem{Parameters}
\sphinxAtStartPar
\sphinxstyleliteralstrong{\sphinxupquote{num\_start\_cones}} \textendash{} The number of cones at the beginning to recalculate estimates and marginal covariances for. 

\end{description}\end{quote}

\end{fulllineitems}

\index{slam::SLAMEstAndMCov::update\_and\_recalculate\_cone\_proximity (C++ function)@\spxentry{slam::SLAMEstAndMCov::update\_and\_recalculate\_cone\_proximity}\spxextra{C++ function}}

\begin{fulllineitems}
\phantomsection\label{\detokenize{isam2:_CPPv4N4slam14SLAMEstAndMCov37update_and_recalculate_cone_proximityENSt6size_tE}}\phantomsection\label{\detokenize{isam2:_CPPv3N4slam14SLAMEstAndMCov37update_and_recalculate_cone_proximityENSt6size_tE}}\phantomsection\label{\detokenize{isam2:_CPPv2N4slam14SLAMEstAndMCov37update_and_recalculate_cone_proximityENSt6size_tE}}\phantomsection\label{\detokenize{isam2:slam::SLAMEstAndMCov::update_and_recalculate_cone_proximity__std::s}}
\pysigstartsignatures
\pysigstartmultiline
\pysiglinewithargsret
{\phantomsection\label{\detokenize{isam2:classslam_1_1SLAMEstAndMCov_1a70a19ef2eb1f9aa30b112d7993881fa1}}\DUrole{kt}{void}\DUrole{w}{ }\sphinxbfcode{\sphinxupquote{\DUrole{n}{update\_and\_recalculate\_cone\_proximity}}}}
{\DUrole{n}{std}\DUrole{p}{::}\DUrole{n}{size\_t}\DUrole{w}{ }\DUrole{n}{\DUrole{sig-param}{pivot}}}
{}
\pysigstopmultiline
\pysigstopsignatures
\sphinxAtStartPar
Recalculates the cone estimates and the marginal covariance matrices for the cones in the look radius of the pivot cone. 

\sphinxAtStartPar
For some look\_radius, calculate estimates and marginal covariances for the cones with IDs in {[}pivot \sphinxhyphen{} look\_radius, pivot + look\_radius{]} and {[}pivot, pivot + look\_radius{]}.

\sphinxAtStartPar

For some look\_radius, calculate estimates and marginal covariances for the cones with IDs in {[}pivot \sphinxhyphen{} look\_radius, pivot + look\_radius{]} and {[}pivot, pivot + look\_radius{]}. Still checks if you are in bounds of the IDs.
\begin{quote}\begin{description}
\sphinxlineitem{Parameters}\begin{itemize}
\item {} 
\sphinxAtStartPar
\sphinxstyleliteralstrong{\sphinxupquote{pivot}} \textendash{} The ID of the pivot cone

\item {} 
\sphinxAtStartPar
\sphinxstyleliteralstrong{\sphinxupquote{pivot}} \textendash{} The ID of the pivot cone 

\end{itemize}

\end{description}\end{quote}

\end{fulllineitems}

\index{slam::SLAMEstAndMCov::update\_with\_old\_cones (C++ function)@\spxentry{slam::SLAMEstAndMCov::update\_with\_old\_cones}\spxextra{C++ function}}

\begin{fulllineitems}
\phantomsection\label{\detokenize{isam2:_CPPv4N4slam14SLAMEstAndMCov21update_with_old_conesERKNSt6vectorINSt6size_tEEE}}\phantomsection\label{\detokenize{isam2:_CPPv3N4slam14SLAMEstAndMCov21update_with_old_conesERKNSt6vectorINSt6size_tEEE}}\phantomsection\label{\detokenize{isam2:_CPPv2N4slam14SLAMEstAndMCov21update_with_old_conesERKNSt6vectorINSt6size_tEEE}}\phantomsection\label{\detokenize{isam2:slam::SLAMEstAndMCov::update_with_old_cones__std::vector:std::s:CR}}
\pysigstartsignatures
\pysigstartmultiline
\pysiglinewithargsret
{\phantomsection\label{\detokenize{isam2:classslam_1_1SLAMEstAndMCov_1a594aabd0f7b9b06d7420fa937ac6113f}}\DUrole{kt}{void}\DUrole{w}{ }\sphinxbfcode{\sphinxupquote{\DUrole{n}{update\_with\_old\_cones}}}}
{\DUrole{k}{const}\DUrole{w}{ }\DUrole{n}{std}\DUrole{p}{::}\DUrole{n}{vector}\DUrole{p}{\textless{}}\DUrole{n}{std}\DUrole{p}{::}\DUrole{n}{size\_t}\DUrole{p}{\textgreater{}}\DUrole{w}{ }\DUrole{p}{\&}\DUrole{n}{\DUrole{sig-param}{old\_cone\_ids}}}
{}
\pysigstopmultiline
\pysigstopsignatures
\sphinxAtStartPar
Recalulates the estiamtes and the marginal covariance matrices for the IDs provided in the vector. 

\sphinxAtStartPar
Updates SLAM state with the set of previously seen cones.

\sphinxAtStartPar

Performs an update using their IDs and cones nearby the earliest, latest, and out\sphinxhyphen{}of\sphinxhyphen{}bounds IDs for local consistency
\begin{quote}\begin{description}
\sphinxlineitem{Parameters}\begin{itemize}
\item {} 
\sphinxAtStartPar
\sphinxstyleliteralstrong{\sphinxupquote{old\_cone\_ids}} \textendash{} The IDs of the cones to recalculate estimates and marginal covariances for

\item {} 
\sphinxAtStartPar
\sphinxstyleliteralstrong{\sphinxupquote{old\_cone\_ids}} \textendash{} Vector of landmark IDs that were matched to current observations 

\end{itemize}

\end{description}\end{quote}

\end{fulllineitems}

\index{slam::SLAMEstAndMCov::update\_with\_new\_cones (C++ function)@\spxentry{slam::SLAMEstAndMCov::update\_with\_new\_cones}\spxextra{C++ function}}

\begin{fulllineitems}
\phantomsection\label{\detokenize{isam2:_CPPv4N4slam14SLAMEstAndMCov21update_with_new_conesENSt6size_tE}}\phantomsection\label{\detokenize{isam2:_CPPv3N4slam14SLAMEstAndMCov21update_with_new_conesENSt6size_tE}}\phantomsection\label{\detokenize{isam2:_CPPv2N4slam14SLAMEstAndMCov21update_with_new_conesENSt6size_tE}}\phantomsection\label{\detokenize{isam2:slam::SLAMEstAndMCov::update_with_new_cones__std::s}}
\pysigstartsignatures
\pysigstartmultiline
\pysiglinewithargsret
{\phantomsection\label{\detokenize{isam2:classslam_1_1SLAMEstAndMCov_1a8263c13c84f96267df4d636293391745}}\DUrole{kt}{void}\DUrole{w}{ }\sphinxbfcode{\sphinxupquote{\DUrole{n}{update\_with\_new\_cones}}}}
{\DUrole{n}{std}\DUrole{p}{::}\DUrole{n}{size\_t}\DUrole{w}{ }\DUrole{n}{\DUrole{sig-param}{num\_new\_cones}}}
{}
\pysigstopmultiline
\pysigstopsignatures
\sphinxAtStartPar
Adds information about the new cones to slam\_est and slam\_mcov. 

\sphinxAtStartPar
Appends new cone estimates and their marginal covariances.

\sphinxAtStartPar
Also updates n\_landmarks appropriately
\begin{quote}\begin{description}
\sphinxlineitem{Parameters}\begin{itemize}
\item {} 
\sphinxAtStartPar
\sphinxstyleliteralstrong{\sphinxupquote{num\_new\_cones}} \textendash{} The number of new cones/estimates/marginal covariances to add

\item {} 
\sphinxAtStartPar
\sphinxstyleliteralstrong{\sphinxupquote{num\_new\_cones}} \textendash{} The number of new cones/estimates/marginal covariances to add 

\end{itemize}

\end{description}\end{quote}

\end{fulllineitems}

\index{slam::SLAMEstAndMCov::calculate\_mdist (C++ function)@\spxentry{slam::SLAMEstAndMCov::calculate\_mdist}\spxextra{C++ function}}

\begin{fulllineitems}
\phantomsection\label{\detokenize{isam2:_CPPv4N4slam14SLAMEstAndMCov15calculate_mdistEN5gtsam6Point2E}}\phantomsection\label{\detokenize{isam2:_CPPv3N4slam14SLAMEstAndMCov15calculate_mdistEN5gtsam6Point2E}}\phantomsection\label{\detokenize{isam2:_CPPv2N4slam14SLAMEstAndMCov15calculate_mdistEN5gtsam6Point2E}}\phantomsection\label{\detokenize{isam2:slam::SLAMEstAndMCov::calculate_mdist__gtsam::Point2}}
\pysigstartsignatures
\pysigstartmultiline
\pysiglinewithargsret
{\phantomsection\label{\detokenize{isam2:classslam_1_1SLAMEstAndMCov_1abf9bea4abd413e68825dbedc81590593}}\DUrole{n}{std}\DUrole{p}{::}\DUrole{n}{vector}\DUrole{p}{\textless{}}\DUrole{kt}{double}\DUrole{p}{\textgreater{}}\DUrole{w}{ }\sphinxbfcode{\sphinxupquote{\DUrole{n}{calculate\_mdist}}}}
{\DUrole{n}{gtsam}\DUrole{p}{::}\DUrole{n}{Point2}\DUrole{w}{ }\DUrole{n}{\DUrole{sig-param}{global\_obs\_cone}}}
{}
\pysigstopmultiline
\pysigstopsignatures
\sphinxAtStartPar
Calculates the Mahalanobis distance between the observed cone and the old cone estimates. 

\sphinxAtStartPar
The distances are calculated using a SIMD method through the Eigen library.

\sphinxAtStartPar

The distances are calculated using a SIMD method through the Eigen library.
\begin{quote}\begin{description}
\sphinxlineitem{Parameters}\begin{itemize}
\item {} 
\sphinxAtStartPar
\sphinxstyleliteralstrong{\sphinxupquote{global\_obs\_cone}} \textendash{} The observed cone that we are data associating in global frame 

\item {} 
\sphinxAtStartPar
\sphinxstyleliteralstrong{\sphinxupquote{global\_obs\_cone}} \textendash{} The observed cone that we are data associating in global frame 

\end{itemize}

\sphinxlineitem{Returns}
\sphinxAtStartPar
std::vector\textless{}double\textgreater{} A vector where the ith element represents the mahalanobis distances

\sphinxlineitem{Returns}
\sphinxAtStartPar
A vector of Mahalanobis distances to each tracked landmark 

\end{description}\end{quote}

\end{fulllineitems}

\index{slam::SLAMEstAndMCov::get\_n\_landmarks (C++ function)@\spxentry{slam::SLAMEstAndMCov::get\_n\_landmarks}\spxextra{C++ function}}

\begin{fulllineitems}
\phantomsection\label{\detokenize{isam2:_CPPv4N4slam14SLAMEstAndMCov15get_n_landmarksEv}}\phantomsection\label{\detokenize{isam2:_CPPv3N4slam14SLAMEstAndMCov15get_n_landmarksEv}}\phantomsection\label{\detokenize{isam2:_CPPv2N4slam14SLAMEstAndMCov15get_n_landmarksEv}}\phantomsection\label{\detokenize{isam2:slam::SLAMEstAndMCov::get_n_landmarks}}
\pysigstartsignatures
\pysigstartmultiline
\pysiglinewithargsret
{\phantomsection\label{\detokenize{isam2:classslam_1_1SLAMEstAndMCov_1aff06a557d9b8544328e5baa2137f46a1}}\DUrole{n}{std}\DUrole{p}{::}\DUrole{n}{size\_t}\DUrole{w}{ }\sphinxbfcode{\sphinxupquote{\DUrole{n}{get\_n\_landmarks}}}}
{}
{}
\pysigstopmultiline
\pysigstopsignatures
\sphinxAtStartPar
Gets the number of landmarks in the SLAM estimates. 
\begin{quote}\begin{description}
\sphinxlineitem{Returns}
\sphinxAtStartPar
Number of landmarks 

\end{description}\end{quote}

\end{fulllineitems}

\index{slam::SLAMEstAndMCov::get\_all\_est (C++ function)@\spxentry{slam::SLAMEstAndMCov::get\_all\_est}\spxextra{C++ function}}

\begin{fulllineitems}
\phantomsection\label{\detokenize{isam2:_CPPv4N4slam14SLAMEstAndMCov11get_all_estEv}}\phantomsection\label{\detokenize{isam2:_CPPv3N4slam14SLAMEstAndMCov11get_all_estEv}}\phantomsection\label{\detokenize{isam2:_CPPv2N4slam14SLAMEstAndMCov11get_all_estEv}}\phantomsection\label{\detokenize{isam2:slam::SLAMEstAndMCov::get_all_est}}
\pysigstartsignatures
\pysigstartmultiline
\pysiglinewithargsret
{\phantomsection\label{\detokenize{isam2:classslam_1_1SLAMEstAndMCov_1ad95f4185c4dd8c28640a8bade5f77436}}\DUrole{n}{std}\DUrole{p}{::}\DUrole{n}{vector}\DUrole{p}{\textless{}}\DUrole{n}{gtsam}\DUrole{p}{::}\DUrole{n}{Point2}\DUrole{p}{\textgreater{}}\DUrole{w}{ }\sphinxbfcode{\sphinxupquote{\DUrole{n}{get\_all\_est}}}}
{}
{}
\pysigstopmultiline
\pysigstopsignatures
\sphinxAtStartPar
Gets the SLAM cone estimates. 
\begin{quote}\begin{description}
\sphinxlineitem{Returns}
\sphinxAtStartPar
A vector of 2D points representing the estimated landmark positions 

\end{description}\end{quote}

\end{fulllineitems}

\index{slam::SLAMEstAndMCov::check\_lengths (C++ function)@\spxentry{slam::SLAMEstAndMCov::check\_lengths}\spxextra{C++ function}}

\begin{fulllineitems}
\phantomsection\label{\detokenize{isam2:_CPPv4N4slam14SLAMEstAndMCov13check_lengthsEv}}\phantomsection\label{\detokenize{isam2:_CPPv3N4slam14SLAMEstAndMCov13check_lengthsEv}}\phantomsection\label{\detokenize{isam2:_CPPv2N4slam14SLAMEstAndMCov13check_lengthsEv}}\phantomsection\label{\detokenize{isam2:slam::SLAMEstAndMCov::check_lengths}}
\pysigstartsignatures
\pysigstartmultiline
\pysiglinewithargsret
{\phantomsection\label{\detokenize{isam2:classslam_1_1SLAMEstAndMCov_1a88f032ececf587581276cb1e7022a689}}\DUrole{kt}{bool}\DUrole{w}{ }\sphinxbfcode{\sphinxupquote{\DUrole{n}{check\_lengths}}}}
{}
{}
\pysigstopmultiline
\pysigstopsignatures
\sphinxAtStartPar
An invariant function to check that the lengths between the slam estimates and the marginal covariance matrices are the same. 
\begin{quote}\begin{description}
\sphinxlineitem{Returns}
\sphinxAtStartPar
true if the lengths are the same 

\sphinxlineitem{Returns}
\sphinxAtStartPar
false if the lengths are not the same 

\end{description}\end{quote}

\end{fulllineitems}

\index{slam::SLAMEstAndMCov::get\_landmark\_symbol (C++ function)@\spxentry{slam::SLAMEstAndMCov::get\_landmark\_symbol}\spxextra{C++ function}}

\begin{fulllineitems}
\phantomsection\label{\detokenize{isam2:_CPPv4N4slam14SLAMEstAndMCov19get_landmark_symbolEi}}\phantomsection\label{\detokenize{isam2:_CPPv3N4slam14SLAMEstAndMCov19get_landmark_symbolEi}}\phantomsection\label{\detokenize{isam2:_CPPv2N4slam14SLAMEstAndMCov19get_landmark_symbolEi}}\phantomsection\label{\detokenize{isam2:slam::SLAMEstAndMCov::get_landmark_symbol__i}}
\pysigstartsignatures
\pysigstartmultiline
\pysiglinewithargsret
{\phantomsection\label{\detokenize{isam2:classslam_1_1SLAMEstAndMCov_1a493e5cde8f32c6a4c38f4aadfd023a85}}\DUrole{n}{gtsam}\DUrole{p}{::}\DUrole{n}{Symbol}\DUrole{w}{ }\sphinxbfcode{\sphinxupquote{\DUrole{n}{get\_landmark\_symbol}}}}
{\DUrole{kt}{int}\DUrole{w}{ }\DUrole{n}{\DUrole{sig-param}{id}}}
{}
\pysigstopmultiline
\pysigstopsignatures
\sphinxAtStartPar
Get the landmark symbol object. 
\begin{quote}\begin{description}
\sphinxlineitem{Parameters}\begin{itemize}
\item {} 
\sphinxAtStartPar
\sphinxstyleliteralstrong{\sphinxupquote{id}} \textendash{} 

\item {} 
\sphinxAtStartPar
\sphinxstyleliteralstrong{\sphinxupquote{id}} \textendash{} Landmark ID 

\end{itemize}

\sphinxlineitem{Returns}
\sphinxAtStartPar
gtsam::Symbol

\sphinxlineitem{Returns}
\sphinxAtStartPar
Corresponding GTSAM symbol 

\end{description}\end{quote}

\end{fulllineitems}


\end{sphinxuseclass}
\end{fulllineitems}


\sphinxstepscope


\section{Glossary}
\label{\detokenize{glossary:glossary}}\label{\detokenize{glossary::doc}}\begin{description}
\sphinxlineitem{SLAM\index{SLAM@\spxentry{SLAM}|spxpagem}\phantomsection\label{\detokenize{glossary:term-SLAM}}}
\sphinxAtStartPar
Simultaneous Localization and Mapping. A method used by autonomous systems to build a map of an unknown environment while simultaneously keeping track of the system’s own position within that environment.

\sphinxlineitem{GTSAM\index{GTSAM@\spxentry{GTSAM}|spxpagem}\phantomsection\label{\detokenize{glossary:term-GTSAM}}}
\sphinxAtStartPar
Georgia Tech Smoothing and Mapping (GTSAM) is a C++ library, with Python bindings, that implements smoothing and mapping using factor graphs. It is widely used in robotics and computer vision for tasks such as SLAM (Simultaneous Localization and Mapping) and Structure from Motion (SfM), enabling efficient estimation of trajectories and landmark positions through probabilistic inference.

\sphinxlineitem{iSAM2\index{iSAM2@\spxentry{iSAM2}|spxpagem}\phantomsection\label{\detokenize{glossary:term-iSAM2}}}
\sphinxAtStartPar
A specific algorithm for SLAM (Incremental Smoothing and Mapping, version 2). It uses a factor graph representation and performs efficient incremental updates to optimize the vehicle’s pose and the map of the environment.

\sphinxlineitem{SVM\index{SVM@\spxentry{SVM}|spxpagem}\phantomsection\label{\detokenize{glossary:term-SVM}}}
\sphinxAtStartPar
Support Vector Machine. A supervised machine learning algorithm used for classification and regression tasks. It finds the optimal hyperplane that separates data into distinct classes with maximum margin.

\sphinxlineitem{Factor Graph\index{Factor Graph@\spxentry{Factor Graph}|spxpagem}\phantomsection\label{\detokenize{glossary:term-Factor-Graph}}}
\sphinxAtStartPar
A bipartite graph consisting of variable nodes and factor nodes.
\begin{itemize}
\item {} 
\sphinxAtStartPar
\sphinxstylestrong{Variable nodes} represent unknowns, such as the vehicle pose or landmark positions.

\item {} 
\sphinxAtStartPar
\sphinxstylestrong{Factor nodes} represent probabilistic constraints between variables, such as measurements or priors.

\end{itemize}

\sphinxAtStartPar
The factor graph encodes the full probabilistic model used for SLAM.

\sphinxlineitem{Joint probability distribution\index{Joint probability distribution@\spxentry{Joint probability distribution}|spxpagem}\phantomsection\label{\detokenize{glossary:term-Joint-probability-distribution}}}
\sphinxAtStartPar
A statistical measure that gives the probability of two or more random variables occurring simultaneously. In SLAM, it represents the combined likelihood of vehicle poses and landmark positions given all observations and constraints.

\sphinxlineitem{Joint probabilistic distribution\index{Joint probabilistic distribution@\spxentry{Joint probabilistic distribution}|spxpagem}\phantomsection\label{\detokenize{glossary:term-Joint-probabilistic-distribution}}}
\sphinxAtStartPar
See {\hyperref[\detokenize{glossary:term-Joint-probability-distribution}]{\sphinxtermref{\DUrole{xref}{\DUrole{std}{\DUrole{std-term}{joint probability distribution}}}}}}.

\sphinxlineitem{Variable Node\index{Variable Node@\spxentry{Variable Node}|spxpagem}\phantomsection\label{\detokenize{glossary:term-Variable-Node}}}
\sphinxAtStartPar
A node in the factor graph representing a variable to be estimated, such as a vehicle pose (\sphinxtitleref{x\_n}) or landmark position (\sphinxtitleref{l\_n}).

\sphinxlineitem{Factor Node\index{Factor Node@\spxentry{Factor Node}|spxpagem}\phantomsection\label{\detokenize{glossary:term-Factor-Node}}}
\sphinxAtStartPar
A node that connects one or more variable nodes and encodes a probabilistic constraint, such as a sensor measurement.

\sphinxlineitem{Prior Factor\index{Prior Factor@\spxentry{Prior Factor}|spxpagem}\phantomsection\label{\detokenize{glossary:term-Prior-Factor}}}
\sphinxAtStartPar
A special type of factor node used to initialize the estimation problem with known values (e.g., the first car pose).

\sphinxlineitem{Bearing\sphinxhyphen{}range factor\index{Bearing\sphinxhyphen{}range factor@\spxentry{Bearing\sphinxhyphen{}range factor}|spxpagem}\phantomsection\label{\detokenize{glossary:term-Bearing-range-factor}}}
\sphinxAtStartPar
A type of measurement used in SLAM that combines the \sphinxstyleemphasis{bearing} (angle) and \sphinxstyleemphasis{range} (distance) between the robot and a landmark. This factor relates the robot’s pose to the landmark’s position and is commonly used in factor graph optimization to improve landmark and pose estimation accuracy.

\sphinxlineitem{Mahalanobis Distance\index{Mahalanobis Distance@\spxentry{Mahalanobis Distance}|spxpagem}\phantomsection\label{\detokenize{glossary:term-Mahalanobis-Distance}}}
\sphinxAtStartPar
A distance metric that accounts for uncertainty in measurements. It measures the number of standard deviations a point is from the mean of a distribution. Used for data association in SLAM.

\sphinxlineitem{Data Association\index{Data Association@\spxentry{Data Association}|spxpagem}\phantomsection\label{\detokenize{glossary:term-Data-Association}}}
\sphinxAtStartPar
The process of matching observed landmarks (e.g., cones) with previously seen landmarks in the map. Essential for consistent mapping and localization.

\sphinxlineitem{IMU\index{IMU@\spxentry{IMU}|spxpagem}\phantomsection\label{\detokenize{glossary:term-IMU}}}
\sphinxAtStartPar
An Inertial Measurement Unit (IMU) is a sensor device that measures and reports a vehicle’s specific force, angular rate, and sometimes magnetic field. It typically contains accelerometers, gyroscopes, and sometimes magnetometers, and is used to estimate orientation, velocity, and motion of the system it is attached to.

\sphinxlineitem{GPS\index{GPS@\spxentry{GPS}|spxpagem}\phantomsection\label{\detokenize{glossary:term-GPS}}}
\sphinxAtStartPar
Global Positioning System (GPS) is a satellite\sphinxhyphen{}based navigation system that provides location and time information anywhere on or near the Earth. It is commonly used in robotics to obtain position estimates for localization and navigation.

\sphinxlineitem{Pose\index{Pose@\spxentry{Pose}|spxpagem}\phantomsection\label{\detokenize{glossary:term-Pose}}}
\sphinxAtStartPar
The position and orientation of the vehicle in the global frame. Often represented as (x, y, θ) in 2D SLAM.

\sphinxlineitem{Quaternion\index{Quaternion@\spxentry{Quaternion}|spxpagem}\phantomsection\label{\detokenize{glossary:term-Quaternion}}}
\sphinxAtStartPar
A 4D representation of 3D orientation that avoids singularities and discontinuities. Consists of four components: (w, x, y, z).

\sphinxlineitem{Twist\index{Twist@\spxentry{Twist}|spxpagem}\phantomsection\label{\detokenize{glossary:term-Twist}}}
\sphinxAtStartPar
A message (in ROS) containing the linear and angular velocity of a vehicle, used for estimating motion.

\sphinxlineitem{Odometry\index{Odometry@\spxentry{Odometry}|spxpagem}\phantomsection\label{\detokenize{glossary:term-Odometry}}}
\sphinxAtStartPar
The use of motion sensors (such as encoders, IMU, or GPS) to estimate the position and orientation of a robot over time.

\sphinxlineitem{Landmark\index{Landmark@\spxentry{Landmark}|spxpagem}\phantomsection\label{\detokenize{glossary:term-Landmark}}}
\sphinxAtStartPar
A static feature in the environment used for localization, such as a cone in Formula Student Driverless.

\sphinxlineitem{ROS (Robot Operating System)\index{ROS (Robot Operating System)@\spxentry{ROS}\spxextra{Robot Operating System}|spxpagem}\phantomsection\label{\detokenize{glossary:term-ROS-Robot-Operating-System}}}
\sphinxAtStartPar
An open\sphinxhyphen{}source framework for building robotic systems. ROS provides tools, libraries, and conventions for writing modular robot software, including message\sphinxhyphen{}passing between processes (nodes), hardware abstraction, and device drivers. In this project, ROS 2 is used to implement and run various nodes such as SLAM, perception, and control.

\sphinxlineitem{ROS 2 Node\index{ROS 2 Node@\spxentry{ROS 2 Node}|spxpagem}\phantomsection\label{\detokenize{glossary:term-ROS-2-Node}}}
\sphinxAtStartPar
A process that performs computation in the ROS 2 framework. In this context, SLAM nodes are responsible for managing factor graphs, receiving sensor data, and publishing localization estimates.

\sphinxlineitem{Rosbag\index{Rosbag@\spxentry{Rosbag}|spxpagem}\phantomsection\label{\detokenize{glossary:term-Rosbag}}}
\sphinxAtStartPar
A file format and tool in ROS used to record and replay messages from topics. Rosbags  allow for offline analysis, debugging, or simulation.

\sphinxlineitem{TBB\index{TBB@\spxentry{TBB}|spxpagem}\phantomsection\label{\detokenize{glossary:term-TBB}}}
\sphinxAtStartPar
(Threading Building Blocks) A C++ library used for parallel programming. It can improve performance in multi\sphinxhyphen{}threaded SLAM systems.

\end{description}



\renewcommand{\indexname}{Index}
\printindex
\end{document}